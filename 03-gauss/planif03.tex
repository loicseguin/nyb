\chapter{Théorème de Gauss}


\section{Équilibre électrostatique}

\marginnote{
  Tremblay \S 2.9

  Lafrance \S 3.5
}

\minisec{Objectif}

\begin{enumerate}
  \item L'étudiant comprendra ce qui se passe avec les charges dans un
    conducteur soumis à un champ électrique.
  \item L'étudiant saura comment déterminer le champ électrique dans un
    conducteur de même que dans une cavité à l'intérieur d'un conducteur.
\end{enumerate}


\minisec{Le champ électrique à l'intérieur du conducteur est nul}

Preuve: s'il ne l'était pas, le champ électrique à l'intérieur exercerait une
force sur les électrons libres. Ils se déplaceraient dans la direction opposée
au champ électrique. On aurait alors une charge $+$ solitaire et une charge $-$
solitaire qui généreraient un champ électrique dans la direction opposée à
celle du champ initial. Le processus continuerait tant et aussi longtemps que
le champ électrique initial n'a pas été complètement neutralisé.



\minisec{Le champ électrique à la surface d'un conducteur est toujours
  perpendiculaire à la surface}

Preuve: si ce n'était pas le cas, on aurait un mouvement des électrons libres
proches de la surface qui finirait par neutraliser la composante tangentielle
du champ.



\minisec{Dans un conducteur chargé, toute la charge se retrouve à la surface
  extérieure}

Preuve: Le champ électrique à l'intérieur doit être nul. Si une charge nette
macroscopique se trouvait à l'intérieur, à proximité de cette charge, le champ
électrique serait non-nul.



\minisec{Dans une cavité à l'intérieur d'un conducteur, le champ électrique est
  nul}

On crée une cavité dans le conducteur. Tout le matériel enlevé est
neutre à cause du paragraphe précédent. Par conséquent, le matériel enlevé
n'exerçait aucune force sur les charges situées à la surface du conducteur.
Comme les charges à la surface étaient à l'équilibre avant d'enlever le
matériel, enlever du matériel qui n'exerçait aucune force ne changera pas leur
état d'équilibre.

Le champ électrique où le matériel a été enlevé était nul avant de faire le
trou. On a enlevé du matériel neutre qui ne contribuait donc pas au champ
électrique. Les charges qui étaient à la surface n'ont pas bougé (argument du
paragraphe précédent). Le champ électrique est déterminé complètement par la
configuration des charges. Par conséquent, le champ électrique dans la cavité
doit encore être nul.


\begin{diapobox}
 Exercices avec une cage de Faraday
\end{diapobox}



\section{Champ électrique dans les diélectriques}

\marginnote{
  Tremblay \S 2.10

  Lafrance \S 3.6
}

\minisec{Constante diélectrique}

Le champ électrique induit dans un diélectrique est en général proportionnel au
champ électrique externe et dans la direction opposée au champ externe
$$\vE_\text{diel} = -\alpha \vE_\text{vide}$$
pour une certaine constante positive $\alpha$. Le champ électrique total est
donc
$$\vE = \vE_\text{vide} + \vE_\text{diel} = (1 - \alpha) \vE_\text{vide}.$$
On définit la \textbf{constante diélectrique} comme
$$\kappa \equiv \frac{1}{1 - \alpha}$$
c'est-à-dire que
$$\vE = \frac{1}{\kappa} \vE_\text{vide}.$$
La constante diélectrique est toujours plus grande ou égale à 1 donc le champ
dans un diélectrique est plus faible que le champ extérieur.


\minisec{Rigidité diélectrique}

Si le champ électrique externe appliqué sur un diélectrique est trop élevé, les
électrons seront arrachés aux atomes et un courant pourra traverser le
diélectrique. C'est ce qu'on appelle une \textbf{décharge}. Chaque matériau est
capable de supporter un champ électrique maximum qu'on appelle la
\textbf{rigidité diélectrique}.

