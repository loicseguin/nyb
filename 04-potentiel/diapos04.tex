\documentclass{beamer}
\beamertemplatenavigationsymbolsempty
\usepackage[french]{babel}
\usepackage{fontspec}
\usepackage{amsmath, amsthm, amsfonts}
\usepackage[separate-uncertainty]{siunitx}
\usepackage{xcolor}
\usepackage{tikz}
\usepackage{tikz-cd}
\usepackage[object=vectorian]{pgfornament}
\usepackage{circuitikz}
\usepackage{hyperref}
\usepackage{caption}
\usepackage{booktabs}
\usepackage{mathtools}
\usepackage{longtable}
\usepackage[version=3]{mhchem}

\tikzset{>=latex,
    negative/.style={draw=teal!70!black, fill=teal!10, thick},
    positive/.style={draw=red, fill=red!10, thick}}
\usetikzlibrary{matrix,calc,decorations.pathreplacing,decorations.pathmorphing,decorations.markings}
\sisetup{locale=FR, per-mode=symbol}

\newcommand{\abs}[1]{\left| #1 \right|}
\newcommand{\rhat}{\vec{\hat{r}}}
\newcommand{\xhat}{\vec{\imath}}
\newcommand{\yhat}{\vec{\jmath}}
\newcommand{\zhat}{\vec{k}}
\newcommand{\real}{\mathbb{R}}
\newcommand{\der}[2]{\frac{\mathrm{d}#1}{\mathrm{d}#2}}
\newcommand{\pder}[2]{\frac{\partial\ #1}{\partial\ #2}}
\newcommand{\dif}{\mathrm{d}}
\newcommand{\ddif}{\,\mathrm{d}}
\newcommand{\grad}{\vec{\nabla}}
\newcommand{\exemple}[1]{\begin{fullwidth}#1\end{fullwidth}}
\newcommand{\norm}[1]{\lVert\ #1\ \rVert}
\newcommand{\vu}{\vec{u}}
\newcommand{\vv}{\vec{v}}
\newcommand{\vr}{\vec{r}}
\newcommand{\va}{\vec{a}}
\newcommand{\vF}{\vec{F}}
\newcommand{\vE}{\vec{E}}
\newcommand{\vB}{\vec{B}}
\newcommand{\vecxyz}[3]{#1 \xhat\ + #2 \yhat\ + #3 \zhat}
\newcommand{\vecxy}[2]{#1 \xhat\ + #2 \yhat}
\newcommand{\coulombcst}{k}
\newcommand{\emf}{\ensuremath{\mathcal{E}}}

% Nice separator line
\newcommand{\sectionline}{
    \noindent
    \begin{center}
        \resizebox{0.5\linewidth}{1ex}
    {{%
    {\begin{tikzpicture}
    \node  (C) at (0,0) {};
    \node (D) at (9,0) {};
    \path (C) to [ornament=85] (D);
    \end{tikzpicture}}}}
    \end{center}
}

\theoremstyle{definition}
\newtheorem*{defn}{Definition}


\usepackage[version=3]{mhchem}

\definecolor{UniBlue}{RGB}{83,121,170}
\setbeamercolor{title}{fg=UniBlue}
\setbeamercolor{frametitle}{fg=UniBlue}
\setbeamercolor{structure}{fg=UniBlue}

\begin{document}

\begin{frame}[t]{Travail effectué par la force électrique}

Classez les situations suivantes en ordre croissant du travail fait par la
force électrique sur la charge $q > 0$.

\begin{center}
  \includegraphics[scale=0.6]{figures/travail-force-electrique.pdf}
\end{center}

\end{frame}


\begin{frame}[t]{Est-ce que la force électrique est conservative?}

Une particule de déplace de l'origine $\vec{r}_0 = \vecxyz{0}{0}{0}$ jusqu'au
point $\vec{r} = \vecxyz{a}{b}{c}$ dans un champ électrique constant $\vec{E} =
E_z \zhat$ où $E_z$ est constant.

Déterminer le travail fait par la force électrique pour les deux trajectoires
suivantes.

\begin{itemize}
  \item Le long de la ligne droite qui relie $\vec{r}_0$ à $\vec{r}$.
  \item Le long de la ligne qui va de $\vec{r}_0$ à $\vecxyz{a}{b}{0}$, puis le
    long de la ligne qui va de $\vecxyz{a}{b}{0}$ à $\vecxyz{a}{b}{c}$.
\end{itemize}

\end{frame}


\begin{frame}[t]{Exemple}

On considère une mince tige métallique cylindrique très longue portant une
charge uniforme.  La densité linéique de charge est de $\lambda =
\SI{0.948}{\micro\coulomb\per\meter}$. Un électron est relâché \SI{35}{mm}
de la tige.

\begin{enumerate}
  \item Calculer la variation d'énergie potentielle du système après que
    l'électron se soit déplacé de \SI{1}{cm}.
  \item Si l'électron était initialement au repos, déterminer sa vitesse à la
    fin de son déplacement.
  \item Tracer un graphique du module du champ électrique généré par la tige en
    fonction de la distance par rapport à son axe.
  \item Tracer un graphique de l'énergie potentielle de l'électron en fonction
    de sa distance par rapport à l'axe de la tige.
\end{enumerate}

\end{frame}


\begin{frame}{Exemple}

Deux grandes plaques métalliques sont maintenues à une différence de potentiel
de \SI{12}{V}. Elles sont séparées d'une distance de \SI{1}{cm}. Un électron
qui se trouve juste à côté d'une des plaques se déplace jusqu'à l'autre plaque.
Si l'électron est initialement immobile, déterminer le module de sa vitesse
lorsqu'il atteint l'autre plaque.

\end{frame}



\begin{frame}{Exemple}

Deux grandes plaques métalliques sont maintenues à une différence de potentiel
de \SI{12}{V}. On augmente la distance entre les plaques. Expliquez ce qui doit
se produire avec la densité surfacique de charge sur les plaques pour que la
différence de potentiel demeure constante.

\end{frame}



\begin{frame}{Exercice sur le potentiel et l'énergie potentielle}

Quatre charges sont placées aux sommets d'un carré de côté $d = \SI{3}{cm}$. Les
charges sont de $q_1 = \SI{1.7}{\micro\coulomb}$, $q_2 = \SI{-3}{\micro\coulomb}$,
$q_3 = \SI{5}{\micro\coulomb}$ et $q_4 = \SI{-4}{\micro\coulomb}$.


\begin{itemize}
  \item Déterminer le potentiel à \SI{2}{cm} en-dessous de la charge $q_1$.
  \item Déterminer l'énergie potentielle de cette distribution de charges.
\end{itemize}

\begin{center}
  \begin{tikzpicture}
    \coordinate (p1) at (0, 0);
    \coordinate (p2) at (2, 0);
    \coordinate (p3) at (2, 2);
    \coordinate (p4) at (0, 2);
    \node[draw, circle] (q1) at (p1) {$q_1$};
    \node[draw, circle] (q2) at (p2) {$q_2$};
    \node[draw, circle] (q3) at (p3) {$q_3$};
    \node[draw, circle] (q4) at (p4) {$q_4$};
    \draw[<->] (q1) -- node[fill=white] {$d$} (q2);
    \draw[<->] (q2) -- node[fill=white] {$d$} (q3);
    \draw[dashed] (q1) -- (q4);
    \draw[dashed] (q3) -- (q4);
  \end{tikzpicture}
\end{center}
\end{frame}



\begin{frame}{Exercice équipotentielles}
On place un électron dans une région de l'espace où les surfaces
équipotentielles sont telles qu'illustrées dans le schéma ci-dessous.

\begin{center}
  \includegraphics[width=5cm]{figures/exercice-equipotentielles.pdf}
\end{center}

\begin{enumerate}
  \item Dans quelle direction est la force que subit l'électron?
  \item Tracez les lignes de champ électrique.
\end{enumerate}
\end{frame}


\end{document}
