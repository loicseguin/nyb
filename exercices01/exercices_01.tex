\documentclass{tufte-handout}

\usepackage{cmbright}
\usepackage[french]{babel}
\usepackage[utf8]{inputenc}
\usepackage[T1]{fontenc}
\usepackage{amsmath, amsthm, amsfonts}
\usepackage[separate-uncertainty]{siunitx}
\usepackage{xcolor}
\usepackage{tikz}
\usepackage{pgfplots}
\usepackage{circuitikz}
\usepackage{hyperref}
\usepackage{booktabs}
\usepackage[version=3]{mhchem}
\pgfplotsset{compat = 1.3}

% Paul Tol's qualitative palette
% ``bright''.https://personal.sron.nl/~pault/#sec:qualitative
\definecolor{tblue}{HTML}{4477AA}
\definecolor{tcyan}{HTML}{66CCEE}
\definecolor{tgreen}{HTML}{228833}
\definecolor{tyellow}{HTML}{CCBB44}
\definecolor{tred}{HTML}{EE6677}
\definecolor{tpurple}{HTML}{AA3377}
\definecolor{tgrey}{HTML}{BBBBBB}


% Default arrow for tikz and style for positive and negative objects.
\tikzset{>=latex,
    negative/.style={draw=teal!70!black, fill=teal!10, thick},
    positive/.style={draw=red, fill=red!10, thick}}
\usetikzlibrary{matrix,calc,decorations.pathreplacing,decorations.pathmorphing,decorations.markings}

% French locale for numbers and negative exponent for units.
\sisetup{locale=FR, per-mode=symbol}

\newcommand{\abs}[1]{\left| #1 \right|}
\newcommand{\rhat}{\vec{\hat{r}}}
\newcommand{\xhat}{\vec{\imath}}
\newcommand{\yhat}{\vec{\jmath}}
\newcommand{\zhat}{\vec{k}}
\newcommand{\real}{\mathbb{R}}
\newcommand{\der}[2]{\frac{\mathrm{d}#1}{\mathrm{d}#2}}
\newcommand{\pder}[2]{\frac{\partial\ #1}{\partial\ #2}}
\newcommand{\dif}{\mathrm{d}}
\newcommand{\ddif}{\,\mathrm{d}}
\newcommand{\grad}{\vec{\nabla}}
\newcommand{\exemple}[1]{\begin{fullwidth}#1\end{fullwidth}}
\newcommand{\norm}[1]{\lVert\ #1\ \rVert}
\newcommand{\vu}{\vec{u}}
\newcommand{\vv}{\vec{v}}
\newcommand{\vr}{\vec{r}}
\newcommand{\va}{\vec{a}}
\newcommand{\vF}{\vec{F}}
\newcommand{\vE}{\vec{E}}
\newcommand{\vB}{\vec{B}}
\newcommand{\vecxyz}[3]{#1 \xhat\ + #2 \yhat\ + #3 \zhat}
\newcommand{\vecxy}[2]{#1 \xhat\ + #2 \yhat}
\newcommand{\coulombcst}{k}
\newcommand{\emf}{\ensuremath{\mathcal{E}}}
\newcommand{\eval}{\SI{1.602e-19}{C}}
\newcommand{\kval}{\SI{8.99e9}{Nm^2 \per C^2}}




% Pour cacher les réponses, utiliser la valeur 0
%\def\reponse{0}
% Pour afficher les réponses, utiliser la valeur 1
\def\reponse{1}


\title{Charges et champs}
\date{}
\author{203-NYB-05 Électricité et magnétisme}


\begin{document}

\maketitle


\section{Fonctionnement d'un tube à rayons X}

Un circuit à haute tension maintient une charge négative sur une plaque à
proximité du filament et une charge positive sur l'anode. Pour une radiographie
typique, on utilise un champ électrique de \SI{500}{\kilo\newton\per\coulomb}
entre la cathode et l'anode. En supposant que la cathode et l'anode sont des
grandes plaques uniformément chargées, déterminez la densité surfacique de
charge sur chacune des deux plaques.

Dans un tube à rayons X, on chauffe un filament en tungstène (semblable à ce
qu'on peut retrouver dans une ampoule à incandescence) en le soumettant à un
courant élevé. La chaleur permet à certains électrons d'être arrachés au
filament lors d'un processus appelé \textbf{thermoionisation}. Le filament se
trouve dans une section du tube qu'on appelle la cathode. Les électrons
arrachés sont accélérés (grâce à un champ électrique dont nous parlerons plus
loin) vers une partie du tube qu'on appelle l'anode. Dans une radiographie
typique, on transfère une charge de \SI{40.0}{\milli\coulomb} de la cathode à
l'anode.

Combien d'électrons sont arrachés à la cathode?

\if\reponse1
  {\color{tblue}
  }
\else
  \vspace{5cm}
\fi


Combien d'électrons en surplus l'anode possède-t-elle après le transfert?

\if\reponse1
  {\color{tblue}
  }
\else
  \vspace{5cm}
\fi

En réalité, les électrons excédentaires ne restent pas à l'anode, ils sont
retournés à la cathode via un circuit à l'extérieur du tube.


\subsection{Premier modèle : la tige comme une charge ponctuelle}

Vous vous dites que le modèle le plus simple est de considérer la tige comme
une charge ponctuelle située en son centre. En utilisant ce modèle, calculez le
champ électrique au point $P$.
\begin{marginfigure}
  \begin{tikzpicture}
    \draw[black!40] (0, -0.1) rectangle (3.5, 0.1);
    \coordinate (P) at (5, 0);
    \node[below] (Pn) at (P) {$P$};
    \fill (P) circle(2pt);
    \foreach \x in {0.2, 0.7, ..., 3.3} {
      \node[black!40] at (\x, 0) {$-$};
    }
    \fill (1.75, 0) circle (2pt);
    \node[below] at (1.75, 0) {$Q$};
  \end{tikzpicture}
\end{marginfigure}

\if\reponse1
  {\color{tblue}
  La distance entre le centre de la tige et le point $P$ est $L/2 + d$. On
  considère que toute la charge $Q$ de la tige est en son centre. Par
  conséquent, le champ électrique est obtenu à partir de l'expression pour le
  champ d'une charge ponctuelle
  \begin{align*}
    E &= \frac{k\abs{Q}}{\left(L/2 + d\right)^2}  \\
      &= \SI{3715}{\newton\per\coulomb}
  \end{align*}
  Le champ pointe vers la tige, donc vers la gauche, parce que la charge de la
  tige est négative.
  }
\else
  \vspace{5cm}
\fi


Est-ce que le résultat que vous obtenez concorde avec le résultat expérimental?

\if\reponse1
  {\color{tblue}
  L'ordre de grandeur est correct, mais avec un écart de plus de
  \SI{20}{\percent}, ce n'est pas un résultat très exact.
  }
\else
\vspace{5cm}
\fi



\subsection{Deuxième modèle : la tige comme deux charges ponctuelles}

Votre premier modèle ne vous satisfait pas. Vous croyez que vous pouvez faire
mieux. Vous pensez à diviser la tige en deux \og sous-tiges \fg\ et à
considérer chacune de ces sous-tiges comme une charge ponctuelle, chacune avec
la moitié de la charge totale de la tige. Une des charge serait à la moitié de
la première sous-tige, l'autre charge à la moitié de la deuxième sous-tige.
\begin{marginfigure}
  \begin{tikzpicture}
    \draw[black!40] (0, -0.1) rectangle (1.75, 0.1);
    \draw[black!40] (1.75, -0.1) rectangle (3.5, 0.1);
    \coordinate (P) at (5, 0);
    \node[below] (Pn) at (P) {$P$};
    \fill (P) circle(2pt);
    \foreach \x in {0.2, 0.7, ..., 3.3} {
      \node[black!40] at (\x, 0) {$-$};
    }
    \fill (0.875, 0) circle (2pt);
    \node[below] at (0.875, 0) {$Q/2$};
    \fill (2.625, 0) circle (2pt);
    \node[below] at (2.625, 0) {$Q/2$};
  \end{tikzpicture}
\end{marginfigure}
En utilisant ce modèle, calculez le champ électrique au point $P$.


\if\reponse1
  {\color{tblue}
  La charge de gauche est à une distance $3L/4 + d$ du point $P$. En utilisant
  l'expression du champ pour une charge ponctuelle, on trouve
  \begin{align*}
    E_\mathrm{gauche} &= \frac{k\abs{Q/2}}{\left(3L/4 + d\right)^2}  \\
      &= \SI{1233.2}{\newton\per\coulomb}
  \end{align*}
  La charge de droite est à une distance $L/4 + d$ du point $P$. En utilisant
  l'expression du champ pour une charge ponctuelle, on trouve
  \begin{align*}
    E_\mathrm{droite} &= \frac{k\abs{Q/2}}{\left(L/4 + d\right)^2}  \\
      &= \SI{3110.7}{\newton\per\coulomb}
  \end{align*}
  Puisque les deux champs pointent dans la même direction (vers la gauche), on
  peut appliquer le principe de superposition directement
  \begin{align*}
    E &= E_\mathrm{gauche} + E_\mathrm{droite}  \\
      &= \SI{4344}{\newton\per\coulomb}
  \end{align*}
  }
\else
  \vspace{10cm}
\fi

Est-ce que le résultat que vous obtenez concorde avec le résultat expérimental?

\if\reponse1
  {\color{tblue}
  C'est beaucoup mieux! L'écart est encore important (environ
  \SI{7}{\percent}), mais on sent qu'on progresse dans la bonne direction.
  }
\else
\vspace{5cm}
\fi




\subsection{Troisième modèle : la tige comme $n$ charges ponctuelles}

Bien heureux de votre succès avec le deuxième modèle, vous décidez de
poursuivre dans cette voie. Vous décidez de \og briser \fg\ la tige en $n$
\og sous-tiges \fg\ chacune de largeur $L/n$ et portant chacune une charge
$Q/n$.
\begin{marginfigure}
  \begin{tikzpicture}
    \foreach \i [evaluate=\i as \x using (0.7 * \i)] in {0, 1, ..., 4} {
      \draw[black!40] (\x, -0.1) rectangle (\x + 0.7, 0.1);
    }
    \coordinate (P) at (5, 0);
    \node[below] (Pn) at (P) {$P$};
    \fill (P) circle(2pt);
    %\foreach \i [evaluate=\i as \x using (2 * \i - 1) * 3.5 / (2 * 5)]
      %in {1, 2, ..., 5} {
      %\fill (\x, 0) circle (2pt);
      %\draw (\x, -0.5) -- (\x, -0.7) node[below] {$x_\i$};
    %}
    \fill (0.35, 0) circle (2pt);
    \draw (0.35, -0.5) -- (0.35, -0.7) node[below] {$x_1$};
    \fill (1.05, 0) circle (2pt);
    \draw (1.05, -0.5) -- (1.05, -0.7) node[below] {$x_2$};
    \node at (1.75, 0) {$\ldots$};
    \draw (1.75, -0.5) -- (1.75, -0.7) node[below] {$\ldots$};
    \fill (2.45, 0) circle (2pt);
    \draw (2.45, -0.5) -- (2.45, -0.7) node[below] {$x_{n - 1}$};
    \fill (3.15, 0) circle (2pt);
    \draw (3.15, -0.5) -- (3.15, -0.7) node[below] {$x_n$};
    \draw[->] (-0.2, -0.5) -- (5.5, -0.5) node[below] {$x$};
    \draw (0, -0.5) -- (0, -0.7) node[below] {$0$};
    \draw (5, -0.5) -- (5, -0.7) node[below] {$x_p$};
  \end{tikzpicture}
\end{marginfigure}
Pour vous aider dans vos calculs, vous définissez un système d'axe avec l'axe
des $x$ qui pointe vers la droite. Vous placez l'origine à l'extrémité gauche
de la tige. Pourquoi n'avez-vous pas vraiment besoin d'un axe $y$ ou $z$ dans
cette situation?

\if\reponse1
  {\color{tblue}
    Le champ généré par la tige au point $P$ pointe vers la gauche parce que le
    point $P$ est aligné avec l'axe de la tige et que la tige est négative. Il
    n'y a aucune composante du champ électrique dans une direction autre que
    celle de l'axe $x$.
  }
\else
  \vspace{1cm}
\fi


Quelle est la position du point $P$, $x_p$?

\if\reponse1
  {\color{tblue}
    Le point $P$ est à une distance $d$ à droite de la fin de la tige de
    longueur $L$. Puisque l'origine est à gauche de la tige, $x_p = L + d =
    \SI{16}{\centi\meter}$.
  }
\else
  \vspace{1cm}
\fi

Vous décidez d'appeler $x_1, x_2, \ldots, x_n$ les positions des différents
morceaux de charge $Q/n$. Écrivez une expression qui vous permet de calculer la
position du morceau de charge $i$, $x_i$.

\if\reponse1
  {\color{tblue}
    \begin{align*}
      x_{i} = (i - 1) \frac{L}{n} + \frac{1}{2}\frac{L}{n}
    \end{align*}
  }
\else
  \vspace{1cm}
\fi


Écrivez l'expression qui permet de calculer la composante $x$ du champ
électrique produit par le morceau de charge $i$.

\if\reponse1
  {\color{tblue}
    \begin{align*}
      E_{ix} = \frac{k Q/n}{(L + d - x_i)^2}
    \end{align*}
  }
\else
  \vspace{1cm}
\fi

Appliquez le principe de superposition pour obtenir une expression pour la
composante $x$ du champ électrique total.

\if\reponse1
  {\color{tblue}
    \begin{align*}
      E_{x} = \sum_i \frac{k Q/n}{(L + d - x_i)^2}
    \end{align*}
  }
\else
  \vspace{1cm}
\fi


En combinant cette expression avec celle qui donne les valeurs de $x_i$, vous
pouvez calculer le champ électrique pour n'importe quel nombre de morceaux.
C'est un peu laborieux, mais un ordinateur est l'outil parfait pour faire ce
genre de calculs répétitifs. Le tableau ci-contre montre les valeurs de champ
obtenu pour différentes valeurs de $n$ jusqu'à $20$.

\marginnote{
  \begin{center}
    \begin{tabular}{SS}
      \toprule
      $n$    &  {$E_x$ (\si{\newton\per\coulomb})}  \\
      \midrule
       1    &   -3714.9  \\
       2    &   -4343.9  \\
       3    &   -4516.8  \\
       4    &   -4585.4  \\
       5    &   -4619.1  \\
       6    &   -4637.9  \\
       7    &   -4649.4  \\
       8    &   -4657.0  \\
       9    &   -4662.3  \\
      10    &   -4666.0  \\
      11    &   -4668.8  \\
      12    &   -4671.0  \\
      13    &   -4672.6  \\
      14    &   -4674.0  \\
      15    &   -4675.0  \\
      16    &   -4675.9  \\
      17    &   -4676.6  \\
      18    &   -4677.2  \\
      19    &   -4677.8  \\
      20    &   -4678.2  \\
      \bottomrule
    \end{tabular}
  \end{center}
}

En regardant les valeurs dans ce tableau, que remarquez-vous qu'il se passe à
mesure que les valeurs de $n$ augmente? Voyez-vous une tendance intéressante?

\if\reponse1
  {\color{tblue}
    La composante $x$ du champ électrique semble se stabiliser à une valeur
    proche de \SI{-4680}{\newton\per\coulomb}. Les mathématiciens diraient que
    la suite semble \textbf{converger} vers cette valeur. Probablement que si
    on continuait les calculs avec des valeurs de plus en plus grandes de $n$,
    on arriverait à une valeur bien spécifique.
  }
\else
  \vspace{1cm}
\fi

Est-ce que le champ électrique obtenu avec $20$ morceaux concorde avec celui
obtenu expérimentalement?

\if\reponse1
  {\color{tblue}
    Oui! Compte tenu de l'incertitude sur la valeur expérimentale, les deux
    valeurs sont égales.
  }
\else
  \vspace{1cm}
\fi


\subsection{Quatrième modèle : la tige comme une infinité de charges
  ponctuelles}

Vous remarquez que plus $n$ devient grand, plus les \og sous-tiges \fg\ sont
petites et plus leur charge est petite. Vous vous retrouvez donc à additionner
un très grand nombre de contributions qui sont chacune très petites. Un ami
vous suggère d'utiliser la notation $dQ$ pour représenter la valeur de charge
(le $Q / n$) lorsque le nombre de morceaux tend vers l'infini. Il vous suggère
aussi de remplacer le symbole usuel de sommation $\sum$ par un S stylisé
$\int$. Vous voyez, votre ami est un mathématicien, et ces gens sont réputés
pour leur appréciation des notations compliquées. En utilisant cette nouvelle
notation, récrivez l'expression pour la composante $x$ du champ électrique au
point $P$ en supposant que le nombre de morceaux tend vers l'infini.

\if\reponse1
  {\color{tblue}
    \begin{align*}
      E_{x} = \int \frac{k dQ}{(L + d - x)^2}
    \end{align*}

    Dans cette expression, le $x$ est la position d'un des petits morceaux.
  }
\else
  \vspace{1cm}
\fi

Dans cette expression, qu'on appelle une intégrale indéfinie, on ne spécifie
pas quel est le domaine d'intégration, c'est-à-dire la région de l'espace sur
laquelle on fait la somme. Évidemment, vous êtes en train d'additionner les
contributions des différents morceaux de la tige, vous devez donc faire
l'intégrale sur toute la tige.

\if\reponse1
  {\color{tblue}
    \begin{align*}
      E_{x} = \int_\mathrm{tige} \frac{k dQ}{(L + d - x)^2}
    \end{align*}
  }
\else
  \vspace{1cm}
\fi

On appelle une des petites sous-tiges un \textbf{élément} de tige. La
longueur d'une des sous-tige est $dx$ et la tige est uniformément chargée,
de telle sorte que la charge par unité de longueur est $\lambda$. Vous pouvez
récrire l'élément de charge $dQ$ en fonction de l'élément de longueur $dx$ et
de la densité linéique de charge $\lambda$.

\if\reponse1
  {\color{tblue}
    \begin{align*}
      dQ = \lambda dx
    \end{align*}
  }
\else
  \vspace{1cm}
\fi

En remplaçant dans l'expression du champ électrique et en prenant soin de bien
indiquer les bornes d'intégration pour que la variable $x$ couvre toute la
tige, vous obtenez une intégrale définie que vous êtes capables d'évaluer grâce
à vos connaissances en calcul intégral. Faites le calcul et vérifiez si la
valeur numérique obtenue concorde avec la valeur expérimentale.

\if\reponse1
  {\color{tblue}
    \begin{align*}
      E_{x} &= \int_\mathrm{tige} \frac{k \lambda dx}{(L + d - x)^2} \\
            &= \int_0^L \frac{k \lambda dx}{(L + d - x)^2} \\
            &= k \lambda \int_0^L \frac{dx}{(L + d - x)^2} \\
            &= k \lambda \int_{L + d}^d \frac{-du}{u^2} &
              \text{où } u = L + d - x \text{ donc } du = -dx  \\
            &= k \lambda \left[ \frac{1}{u} \right]_{L + d}^d  \\
            &= k \lambda \left( \frac{1}{d} - \frac{1}{L + d} \right)
    \end{align*}
    La densité de charge est $\lambda = Q/L = \SI{-5}{\nano\coulomb} /
    \SI{10}{\centi\meter} = \SI{-5e-8}{\coulomb\per\meter}$. On trouve donc
    \begin{align*}
      E_x  &= \SI{8.99e9}{\newton\meter\squared\per\coulomb\squared}
              \left( \SI{-5e8}{\coulomb} \right) \left(
                \frac{1}{\SI{6}{\centi\meter}} - \frac{1}{\SI{10}{\centi\meter}
                  + \SI{6}{\centi\meter}} \right)  \\
           &= \SI{-4682.3}{\newton\per\meter}
    \end{align*}
    Donc $\vE = \SI{-4682.3}{\newton\per\meter}\xhat$.
  }
\else
  \vspace{10cm}
\fi

Si on est très loin du fil, alors le fil ressemble de plus en plus à un point.
Vous vous attendez donc à ce que le champ du fil se rapproche de celui d'un
point. Montrez que c'est le cas en utilisant l'expression symbolique que vous
avez obtenue pour la composante $x$ du champ.

\if\reponse1
  {\color{tblue}
    \begin{align*}
      E_{x} &= k \lambda \left( \frac{1}{d} - \frac{1}{L + d} \right)  \\
            &= k \lambda  \frac{L}{d(L + d)}
    \end{align*}
    Si on est très loin de la tige, alors $d \gg L$ et $L + d \approx d$ d'où
    \begin{align*}
      E_{x} &= k \lambda \left( \frac{1}{d} - \frac{1}{L + d} \right)  \\
            &= k \lambda  \frac{L}{d^2}  \\
            &= \frac{kQ}{d^2}
    \end{align*}

  }
\else
  \vspace{1cm}
\fi







%\section{Champ électrique au-dessus d'une tige uniformément chargée}


%Considérez une tige de longueur $L$, portant une densité linéique de charge
%uniforme $\lambda$. On cherche le champ électrique à une distance $D$ au-dessus
%du centre de la tige.

%\begin{figure}
  %\begin{tikzpicture}
    %\draw (-3, 0) -- (3, 0);
    %\draw (-3, -0.1) -- (3, -0.1);
    %\coordinate (P) at (0, 3);
    %\coordinate (O) at (0, 0);
    %\coordinate (r) at (2, 0);
    %\node[anchor=west] (Pn) at (P) {$P$};
    %\fill (P) circle(2pt);
    %\draw[<->] (O) -- node[fill=white] {$D$} ($(P) - (0, 0.08)$);
    %\fill ($(r) + (0.1, 0)$) rectangle ++(-0.2, -0.1);
    %\draw[dashed] (r) -- node[right] {$r$} (P);
    %\draw ($(P) - (0, 0.5)$) arc (-90:{-atan(3/2}:0.5);
    %\node at ($(P) + (0.2, -0.7)$) {$\theta$};
    %\draw (0, -0.1) -- (0, -0.3) node[below] {$0$};
    %\draw (r) -- ++(0, -0.3) node[below] {$x$};
    %\draw[->] (-2.8, 1.5) -- (-2, 1.5) node[below] {$x$};
    %\draw[->] (-2.8, 1.5) -- (-2.8, 2.5) node[left] {$y$};
  %\end{tikzpicture}
%\end{figure}


%\begin{enumerate}
  %\item Faites un schéma de la situation en vous inspirant du schéma ci-contre.
    %Identifiez un élément de charge $dq$ le long de la tige. Tracez le champ
    %électrique $d\vec{E}$ produit par cet élément de charge.
  %\item Trouvez une expression pour la charge en fonction de la densité
    %linéique de charge et de la longueur de l'élément de charge.
  %\item En utilisant la symétrie de la situation, déduisez la valeur de la
    %composante horizontale du champ électrique net.
  %\item L'élément de charge est suffisamment petit pour se comporter comme une
    %charge ponctuelle. Montrez que l'expression de la composante $y$ du champ
    %électrique au point $P$ est
    %\[
      %dE_y = \frac{k \lambda dx}{r^2} \cos \theta.
    %\]
  %\item L'expression précédente contient trois variables $x$, $r$ et $\theta$.
    %Pour être en mesure de calculer l'intégrale, vous devez exprimer le champ
    %électrique en terme d'une seule variable. Ici, vous utiliserez $\theta$.
    %Exprimez $r$ en fonction de $D$ et $\theta$. Exprimez $x$ en fonction de
    %$\theta$ puis dérivez pour trouver $dx$ en fonction de $d\theta$
    %\marginpar{Rappelez-vous que la dérivée de $\tan \theta$ est $\sec^2 \theta$}.
    %Vous devriez obtenir
    %\[
      %dE_y = \frac{k \lambda d\theta}{D} \cos \theta.
    %\]
  %\item Appliquez le principe de superposition pour obtenir une intégrale
    %donnant la composante $y$ du champ électrique net. Les bornes d'intégration
    %devraient être choisies de telle sorte que vous couvrez la tige au complet.

  %\begin{figure}
    %Indice pour les bornes d'intégration:

    %\begin{tikzpicture}
      %\draw (-3, 0) -- (3, 0);
      %\draw (-3, -0.1) -- (3, -0.1);
      %\coordinate (P) at (0, 3);
      %\coordinate (O) at (0, 0);
      %\coordinate (r) at (3, 0);
      %\node[anchor=west] (Pn) at (P) {$P$};
      %\fill (P) circle(2pt);
      %\draw[<->] (O) -- node[fill=white] {$D$} ($(P) - (0, 0.08)$);
      %\fill ($(r) + (0.1, 0)$) rectangle ++(-0.2, -0.1);
      %\draw[dashed] (r) -- node[right] {$\sqrt{L^2/4 + D^2}$} (P);
      %\draw ($(P) - (0, 0.5)$) arc (-90:{-atan(3/3}:0.5);
      %\node at ($(P) + (0.2, -0.7)$) {$\theta$};
      %\draw (0, -0.1) -- (0, -0.3) node[below] {$0$};
      %\draw (r) -- ++(0, -0.3) node[below] {$L/2$};
    %\end{tikzpicture}
  %\end{figure}

  %\item Évaluez l'intégrale. Vous devriez trouver
    %\[
      %\vec{E} = \frac{k \lambda}{D} \frac{L}{\sqrt{L^2 / 4 + D^2}} \yhat.
    %\]

  %\item Montrez que lorsque la distance $D$ est beaucoup plus grande que la
    %longueur de la tige, le champ électrique produit par la tige se rapproche
    %de celui d'une charge ponctuelle :
    %\[
      %\vec{E} = \frac{k\lambda L}{D^2} \yhat.
    %\]

  %\item Montrez que lorsque la tige est très longue ($L$ est beaucoup plus
    %grand que la distance $D$), le champ électrique produit par la tige est
    %donné par
    %\[
      %\vec{E} = \frac{2k\lambda }{D} \yhat.
    %\]

%\end{enumerate}


%\section{Champ électrique d'un arc uniformément chargé}

%On considère un arc de cercle de rayon $R$ qui couvre un angle $\alpha$ et
%porte une densité linéique de charge $\lambda$.  Quel est le champ électrique
%au centre du cercle?

%\begin{figure}
  %\begin{tikzpicture}
    %\draw[decorate, decoration={
      %markings, mark=between positions 0.1 and 1 step 6mm with {
        %\node (0, 0) {$+$};
    %}}] (-40:2.5) arc (-40:40:2.5);
    %\draw (-40:2.7) arc (-40:40:2.7);
    %\draw (-40:2.3) arc (-40:40:2.3);
    %\draw[fill=black] (0, 0) circle (1pt);
    %\draw (0, 0) -- node[fill=white] {$R$} ++(40:2.3);
    %\draw (0, 0) -- ++(-40:2.3);
    %\draw (-40:0.3) arc (-40:40:0.3);
    %\draw[dashed, <-] (-1, 0) node[below] {$x$} -- (3, 0);
    %\draw[fill=black!20] (-30:2.3) -- (-30:2.7) arc (-30:-25:2.7) -- (-25:2.3) -- cycle;
    %\draw[dashed] (-27.5:2.3) -- (0, 0);
    %\draw (0:1) arc (0:-27.5:1);
    %\draw[->|] (-34:2.8) -- (-30:2.8);
    %\draw[->|] (-21:2.8) -- (-25:2.8);
    %\node at (-27:3.2) {$Rd\theta$};
    %\node at (-14:1.3) {$\theta$};
    %\node[fill=white] at (0:0.6) {$\alpha$};
  %\end{tikzpicture}
%\end{figure}

%\begin{enumerate}
  %\item En utilisant un argument de symétrie, montrez que le champ électrique
    %net n'a qu'une composante parallèle à l'axe des $x$.
  %\item Montrez que la composante $x$ du champ électrique produit par un
    %élément de charge $dq$ est donné par
    %\[
      %\dif E_x =  \frac{k\lambda R \ddif \theta}{R^2} \cos \theta.
    %\]
  %\item Appliquez le principe de superposition pour trouver le champ électrique
    %résultant
    %\[
      %\vec{E} = \frac{2k\lambda}{R} \sin(\alpha / 2) \xhat.
    %\]
  %\item À partir du résultat précédent, montrez que le champ électrique au
    %centre d'un anneau uniformément chargé est nul.
%\end{enumerate}


\end{document}
