\documentclass{tufte-handout}

\usepackage{cmbright}
\usepackage[french]{babel}
\usepackage[utf8]{inputenc}
\usepackage[T1]{fontenc}
\usepackage{amsmath, amsthm, amsfonts}
\usepackage[separate-uncertainty]{siunitx}
\usepackage{xcolor}
\usepackage{tikz}
\usepackage{pgfplots}
\usepackage{circuitikz}
\usepackage{hyperref}
\usepackage{booktabs}
\usepackage[version=3]{mhchem}
\pgfplotsset{compat = 1.3}

% Paul Tol's qualitative palette
% ``bright''.https://personal.sron.nl/~pault/#sec:qualitative
\definecolor{tblue}{HTML}{4477AA}
\definecolor{tcyan}{HTML}{66CCEE}
\definecolor{tgreen}{HTML}{228833}
\definecolor{tyellow}{HTML}{CCBB44}
\definecolor{tred}{HTML}{EE6677}
\definecolor{tpurple}{HTML}{AA3377}
\definecolor{tgrey}{HTML}{BBBBBB}


% Default arrow for tikz and style for positive and negative objects.
\tikzset{>=latex,
    negative/.style={draw=teal!70!black, fill=teal!10, thick},
    positive/.style={draw=red, fill=red!10, thick}}
\usetikzlibrary{matrix,calc,decorations.pathreplacing,decorations.pathmorphing,decorations.markings}

% French locale for numbers and negative exponent for units.
\sisetup{locale=FR, per-mode=symbol}

\newcommand{\abs}[1]{\left| #1 \right|}
\newcommand{\rhat}{\vec{\hat{r}}}
\newcommand{\xhat}{\vec{\imath}}
\newcommand{\yhat}{\vec{\jmath}}
\newcommand{\zhat}{\vec{k}}
\newcommand{\real}{\mathbb{R}}
\newcommand{\der}[2]{\frac{\mathrm{d}#1}{\mathrm{d}#2}}
\newcommand{\pder}[2]{\frac{\partial\ #1}{\partial\ #2}}
\newcommand{\dif}{\mathrm{d}}
\newcommand{\ddif}{\,\mathrm{d}}
\newcommand{\grad}{\vec{\nabla}}
\newcommand{\exemple}[1]{\begin{fullwidth}#1\end{fullwidth}}
\newcommand{\norm}[1]{\lVert\ #1\ \rVert}
\newcommand{\vu}{\vec{u}}
\newcommand{\vv}{\vec{v}}
\newcommand{\vr}{\vec{r}}
\newcommand{\va}{\vec{a}}
\newcommand{\vF}{\vec{F}}
\newcommand{\vE}{\vec{E}}
\newcommand{\vB}{\vec{B}}
\newcommand{\vecxyz}[3]{#1 \xhat\ + #2 \yhat\ + #3 \zhat}
\newcommand{\vecxy}[2]{#1 \xhat\ + #2 \yhat}
\newcommand{\coulombcst}{k}
\newcommand{\emf}{\ensuremath{\mathcal{E}}}
\newcommand{\eval}{\SI{1.602e-19}{C}}
\newcommand{\kval}{\SI{8.99e9}{Nm^2 \per C^2}}




% Pour cacher les réponses, utiliser la valeur 0
%\def\reponse{0}
% Pour afficher les réponses, utiliser la valeur 1
\def\reponse{0}


\title{Charges, champ et potentiel}
\date{}
\author{203-NYB-05 Électricité et magnétisme}


\begin{document}

\maketitle


\section{Fonctionnement d'un tube à rayons X}

Dans un tube à rayons X, un circuit à haute tension maintient une charge
négative sur une plaque appelée la cathode et une charge positive sur l'anode.
Pour y arriver, le circuit haute tension transfère des électrons de l'anode à
la cathode. Pour une radiographie typique, on utilise un champ électrique de
\SI{500}{\kilo\newton\per\coulomb} entre la cathode et l'anode. En supposant
que la cathode et l'anode sont des grandes plaques uniformément chargées,
illustrez, sur la figure ci-contre, le champ électrique entre les deux plaques.
Déterminez la densité surfacique de charge sur chacune des deux plaques.

\begin{marginfigure}
  \begin{center}
  \begin{tikzpicture}
    \draw[rounded corners] (0, 0) rectangle (4, 2.5);
    \draw[ultra thick] (0.5, 0.5) -- (0.5, 2);
    \draw[ultra thick] (3.5, 0.5) -- (3.5, 2);
    \draw (0.5, 1.25) -- ++(-1, 0) -- ++(0, 2) -- ++(5, 0)
      -- ++(0, -2) -- ++(-1, 0);
    \fill[black!10] (.75, 2.75) rectangle (3.25, 3.75);
    \node[align=center] at (2, 3.25) {Circuit\\ haute tension};
    \draw[->] (0.4, 0.6) to[bend right=45] (0, -1) node[below] {Anode};
    \draw[->] (3.6, 0.6) to[bend left=45] (3, -1) node[below] {Cathode};
  \end{tikzpicture}
  \end{center}
\end{marginfigure}

\if\reponse1
  {\color{tblue}
    On sait que chacune des plaques produit un champ uniforme de grandeur
    $\abs{\sigma} / 2\varepsilon_0$. Les densités surfaciques sont les mêmes en
    valeur absolues par le principe de conservation de la charge (en supposant
    que le système est initialement neutre). Entre l'anode et la cathode, les
    deux champs pointent dans la même direction (vers la cathode), donc les
    grandeurs s'additionnent par le principe de superposition
    \begin{align*}
      E = \frac{\abs{\sigma}}{\varepsilon_0}
    \end{align*}
    On connait la grandeur du champ, donc on peut déduire la densité surfacique
    de charge:
    \begin{align*}
      \abs{\sigma} &= \varepsilon_0 E  \\
                   &= \SI{4.427e-6}{\coulomb\per\meter\squared}
    \end{align*}
    La cathode a une charge surfacique de
    \SI{-4.427}{\micro\coulomb\per\meter\squared} et l'anode a une charge
    surfacique de \SI{4.427}{\micro\coulomb\per\meter\squared}.
  }
\else
  \vspace{5cm}
\fi


À proximité de la cathode se trouve un filament de tungstène (semblable à ce
qu'on peut retrouver dans une ampoule à incandescence). En faisant circuler un
courant élevé dans ce filament, on le fait chauffer.  La chaleur permet à
certains électrons d'être arrachés au filament lors d'un processus appelé
\textbf{thermoionisation}. On appelle les électrons ainsi arrachés des
\textbf{thermoélectrons}. Les thermoélectrons sont accélérés vers l'anode grâce
au champ électrique. Dans une radiographie typique, le transfert des
thermoélectrons de la cathode à l'anode correspond à une charge de
\SI{-40.0}{\milli\coulomb}. Combien de thermoélectrons sont arrachés à la
cathode?

\if\reponse1
  {\color{tblue}
    On divise simplement la charge totale par la charge d'un électron
    \begin{align*}
      n &= \frac{\SI{-40.0}{\milli\coulomb}}{-e}  \\
        &= \frac{\SI{-40.0}{\milli\coulomb}}{\SI{-1.602e-19}{\coulomb}}  \\
        &= \num{2.497e17}
    \end{align*}
  }
\else
  \vspace{5cm}
\fi

Un thermoélectron, lorsqu'il est arraché au filament, a une vitesse presque
nulle. Quelle est alors sont énergie cinétique?

\if\reponse1
  {\color{tblue}
    Puisque l'énergie cinétique est définie par $K = \frac{1}{2} mv^2$
    et que la vitesse est nulle, on peut déduire que l'énergie cinétique est
    nulle.
  }
\else
  \vspace{3cm}
\fi

\marginnote[\baselineskip]{
  La masse de l'électron est $m_e = \SI{9.109e-31}{\kilo\gram}$.
}
À partir du champ que vous avez déterminé plus haut, calculez la force qui agit
sur un thermoélectron lorsqu'il est entre la cathode et l'anode. Utilisez cette
force pour déduire l'accélération du thermoélectron.

\if\reponse1
  {\color{tblue}
    On pose que l'axe $x$ va vers l'anode. L'électron subit une force donnée par
    \begin{align*}
      \vF &= -e \vE  \\
          &= \SI{-1.602e-19}{\coulomb} \SI{500}{\kilo\newton\per\coulomb} (-\xhat)  \\
          &= \SI{1.602e-19}{\coulomb} \times \SI{500}{\kilo\newton\per\coulomb} \xhat  \\
          &= \SI{8.01e-14}{\newton} \xhat
    \end{align*}
    Par la deuxième loi de Newton, l'accélération de l'électron est
    \begin{align*}
      \va &= \frac{\vF}{m_e}  \\
          &= \frac{\SI{8.01e-14}{\newton}}{\SI{9.109e-31}{\kilo\gram}} \xhat  \\
          &= \SI{8.794e16}{\meter\per\second\squared} \xhat
    \end{align*}

  }
\else
  \vspace{10cm}
\fi


Dans un tube à rayons X, la distance entre la cathode et l'anode est de
\SI{10.0}{\centi\meter}. Quelle est la vitesse d'un thermoélectron lorsqu'il
est rendu à l'anode?

\if\reponse1
  {\color{tblue}
    Si le champ électrique entre la cathode et l'anode est uniforme, alors
    l'accélération de l'électron est uniforme et sont mouvement est un
    mouvement rectiligne uniforme. On pose que l'origine de l'axe $x$ est à la
    cathode.
    \begin{align*}
      x &= x_0 + v_{0x} + \frac{1}{2} a_x t^2  \\
      x &= \frac{1}{2} a_x t^2    \\
      t &= \sqrt{\frac{2x}{a_x}}
    \end{align*}
    Puisque la vitesse est la dérivée de la position
    \begin{align*}
      v_x &= v_{0x} + a_x t  \\
          &= a_x \sqrt{\frac{2x}{a_x}}  \\
          &= \sqrt{2a_xx}  \\
          &= \sqrt{2 \cdot \SI{8.794e16}{\meter\per\second\squared} \cdot
                     \SI{10}{\centi\meter}}  \\
          &= \SI{1.326e8}{\meter\per\second}  \\
      \vv &= \SI{1.326e8}{\meter\per\second} \xhat
    \end{align*}
    C'est un peu moins de la moitié de la vitesse de la lumière!
  }
\else
  \vspace{10cm}
\fi


Quelle est l'énergie cinétique du thermoélectron lorsqu'il est rendu à l'anode?

\if\reponse1
  {\color{tblue}
    \begin{align*}
      K &= \frac{1}{2} m_e v^2  \\
        &= \SI{8.0103e-15}{\joule}
    \end{align*}
  }
\else
  \vspace{5cm}
\fi

Lorsque l'électron entre en collision avec l'anode, cette énergie cinétique
peut être convertie, en tout ou en partie, en un rayon X.

La force électrique est une force conservative, c'est-à-dire qu'on peut lui
associer une énergie potentielle. On peut fixer la valeur d'énergie potentielle
à l'anode à zero. En utilisant le principe de conservation de l'énergie,
déduisez l'énergie potentielle électrique d'un thermoélectron lorsqu'il est à
la cathode.

\if\reponse1
  {\color{tblue}
    \begin{align*}
      U_\mathrm{cathode} &= K_\mathrm{anode}  \\
        &= \SI{8.0103e-15}{\joule}
    \end{align*}
  }
\else
  \vspace{5cm}
\fi



Le potentiel électrique est défini comme l'énergie potentielle par unité de
charge. Quel et le potentiel électrique à la cathode?

\if\reponse1
  {\color{tblue}
    \begin{align*}
      V_\mathrm{cathode} &= \frac{U}{-e}  \\
        &= \SI{-50.0}{\kilo\volt}
    \end{align*}
  }
\else
  \vspace{5cm}
\fi


\section{Bonus}

\marginnote{
  \[h = \SI{6.626e-34}{\joule\second} \]
}
Lorsqu'un électron entre en collision avec l'anode, il peut perdre toute son
énergie cinétique et cette énergie est convertie en un rayon X. Vous verrez,
dans votre troisième cours de physique, que l'énergie d'une \og particule \fg\
de lumière (un \textbf{photon}), est donnée par
\[E_\mathrm{photon} = hf\]
où $h$ est la constante de Planck et $f$ est la fréquence du photon. Si toute
l'énergie cinétique d'un thermoélectron est convertie en énergie d'un photon,
quelle est la fréquence de ce photon? Est-ce que ça correspond bien à la
fréquence d'un rayon X?

\if\reponse1
  {\color{tblue}
    \begin{align*}
    \end{align*}
  }
\else
  \vspace{3cm}
\fi

En supposant qu'environ \SI{0.1}{\percent} de l'énergie cinétique de tous les
thermoélectrons est transférée à l'anode pour produire des rayons X. Quelle est
l'énergie totale dans le faisceau de rayons X?

\if\reponse1
  {\color{tblue}
    \begin{align*}
    \end{align*}
  }
\else
  \vspace{3cm}
\fi

\marginnote[2\baselineskip]{
  L'unité de dose équivalente est le sievert : $\SI{1}{\sievert} =
  \SI{1}{\joule\per\kilogram}$.
}
On peut calculer la \textbf{dose de radiation équivalente} absorbée par le
patient. La dose équivalente pour les rayons X est définie par l'énergie de
radiation absorbée par unité de masse de la personne qui reçoit la radiation
\[ H = \frac{E}{m} \]
où $E$ représente l'énergie de la radiation reçue. Quelle serait votre dose
équivalente si \SI{1}{\percent} de l'énergie des rayons X était absorbée dans
votre corps lors de la radiographie?

\if\reponse1
  {\color{tblue}
    \begin{align*}
    \end{align*}
  }
\else
  \vspace{3cm}
\fi

Sachant que la dose quotidienne de radiation que vous recevez normalement en
provenance de l'environnement est d'environ \SI{10}{\micro\sievert},
devriez-vous être inquiet des effets néfastes sur votre santé d'une
radiographie?



\end{document}
