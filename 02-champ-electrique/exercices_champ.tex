\documentclass{tufte-handout}
\ifxetex
  \usepackage{fontspec}
\else
  \usepackage[utf8]{inputenc}
  \usepackage[T1]{fontenc}
\fi

\usepackage[french]{babel}
\usepackage{fontspec}
\usepackage{amsmath, amsthm, amsfonts}
\usepackage[separate-uncertainty]{siunitx}
\usepackage{xcolor}
\usepackage{tikz}
\usepackage{tikz-cd}
\usepackage[object=vectorian]{pgfornament}
\usepackage{circuitikz}
\usepackage{hyperref}
\usepackage{caption}
\usepackage{booktabs}
\usepackage{mathtools}
\usepackage{longtable}
\usepackage[version=3]{mhchem}
\usepackage{marginnote}
\usepackage[framemethod=tikz]{mdframed}


% Paul Tol's qualitative palette
% ``bright''.https://personal.sron.nl/~pault/#sec:qualitative
\definecolor{tblue}{HTML}{4477AA}
\definecolor{tcyan}{HTML}{66CCEE}
\definecolor{tgreen}{HTML}{228833}
\definecolor{tyellow}{HTML}{CCBB44}
\definecolor{tred}{HTML}{EE6677}
\definecolor{tpurple}{HTML}{AA3377}
\definecolor{tgrey}{HTML}{BBBBBB}


% Justification for marginnotes.
\renewcommand*{\raggedleftmarginnote}{}
\renewcommand*{\raggedrightmarginnote}{}


% Styles for mdframed environments.
\newmdenv[backgroundcolor=tgreen!10,linecolor=tgreen!30]{reponsebox}
\newmdenv[backgroundcolor=tyellow!10,linecolor=tyellow!30]{diapobox}
\newmdenv[backgroundcolor=tred!10,linecolor=tred!30]{fondamentalbox}

% Default arrow for tikz and style for positive and negative objects.
\tikzset{>=latex,
    negative/.style={draw=teal!70!black, fill=teal!10, thick},
    positive/.style={draw=red, fill=red!10, thick}}
\usetikzlibrary{matrix,calc,decorations.pathreplacing,decorations.pathmorphing,decorations.markings}

% French locale for numbers and negative exponent for units.
\sisetup{locale=FR, per-mode=symbol}

\newcommand{\abs}[1]{\left| #1 \right|}
\newcommand{\rhat}{\vec{\hat{r}}}
\newcommand{\xhat}{\vec{\imath}}
\newcommand{\yhat}{\vec{\jmath}}
\newcommand{\zhat}{\vec{k}}
\newcommand{\real}{\mathbb{R}}
\newcommand{\der}[2]{\frac{\mathrm{d}#1}{\mathrm{d}#2}}
\newcommand{\pder}[2]{\frac{\partial\ #1}{\partial\ #2}}
\newcommand{\dif}{\mathrm{d}}
\newcommand{\ddif}{\,\mathrm{d}}
\newcommand{\grad}{\vec{\nabla}}
\newcommand{\exemple}[1]{\begin{fullwidth}#1\end{fullwidth}}
\newcommand{\norm}[1]{\lVert\ #1\ \rVert}
\newcommand{\vu}{\vec{u}}
\newcommand{\vv}{\vec{v}}
\newcommand{\vr}{\vec{r}}
\newcommand{\va}{\vec{a}}
\newcommand{\vF}{\vec{F}}
\newcommand{\vE}{\vec{E}}
\newcommand{\vB}{\vec{B}}
\newcommand{\vecxyz}[3]{#1 \xhat\ + #2 \yhat\ + #3 \zhat}
\newcommand{\vecxy}[2]{#1 \xhat\ + #2 \yhat}
\newcommand{\coulombcst}{k}
\newcommand{\emf}{\ensuremath{\mathcal{E}}}
\newcommand{\eval}{\SI{1.602e-19}{C}}
\newcommand{\kval}{\SI{8.99e9}{Nm^2 \per C^2}}

% Nice separator line
\newcommand{\sectionline}{
    \noindent
    \begin{center}
        \resizebox{0.5\linewidth}{1ex}
    {{%
    {\begin{tikzpicture}
    \node  (C) at (0,0) {};
    \node (D) at (9,0) {};
    \path (C) to [ornament=85] (D);
    \end{tikzpicture}}}}
    \end{center}
}

\theoremstyle{definition}
\newtheorem*{defn}{Definition}


\usepackage{textpos}

\sisetup{locale=FR, per-mode=symbol}

% This is necessary to fix tufte-handout when compiling with XeLaTeX.
\ifxetex
  \newcommand{\textls}[2][5]{%
    \begingroup\addfontfeatures{LetterSpace=#1}#2\endgroup
  }
  \renewcommand{\allcapsspacing}[1]{\textls[15]{#1}}
  \renewcommand{\smallcapsspacing}[1]{\textls[10]{#1}}
  \renewcommand{\allcaps}[1]{\textls[15]{\MakeTextUppercase{#1}}}
  \renewcommand{\smallcaps}[1]{\smallcapsspacing{\scshape\MakeTextLowercase{#1}}}
  \renewcommand{\textsc}[1]{\smallcapsspacing{\textsmallcaps{#1}}}
\fi


\title{Exercices sur les distributions continues de charges}
\date{}
%\author{203-NYB-05 Électricité et magnétisme}


\begin{document}

\maketitle

\section{Champ électrique au-dessus d'une tige uniformément chargée}

Considérez une tige de longueur $L$, portant une densité linéique de charge
uniforme $\lambda$. On cherche le champ électrique à une distance $D$ au-dessus
du centre de la tige.

\begin{marginfigure}
  \begin{tikzpicture}
    \draw (-3, 0) -- (3, 0);
    \draw (-3, -0.1) -- (3, -0.1);
    \coordinate (P) at (0, 3);
    \coordinate (O) at (0, 0);
    \coordinate (r) at (2, 0);
    \node[anchor=west] (Pn) at (P) {$P$};
    \fill (P) circle(2pt);
    \draw[<->] (O) -- node[fill=white] {$D$} ($(P) - (0, 0.08)$);
    \fill ($(r) + (0.1, 0)$) rectangle ++(-0.2, -0.1);
    \draw[dashed] (r) -- node[right] {$r$} (P);
    \draw ($(P) - (0, 0.5)$) arc (-90:{-atan(3/2}:0.5);
    \node at ($(P) + (0.2, -0.7)$) {$\theta$};
    \draw (0, -0.1) -- (0, -0.3) node[below] {$0$};
    \draw (r) -- ++(0, -0.3) node[below] {$x$};
    \draw[->] (-2.8, 1.5) -- (-2, 1.5) node[below] {$x$};
    \draw[->] (-2.8, 1.5) -- (-2.8, 2.5) node[left] {$y$};
  \end{tikzpicture}
\end{marginfigure}


\begin{enumerate}
  \item Faites un schéma de la situation en vous inspirant du schéma ci-contre.
    Identifiez un élément de charge $dq$ le long de la tige. Tracez le champ
    électrique $d\vec{E}$ produit par cet élément de charge.
  \item Trouvez une expression pour la charge en fonction de la densité
    linéique de charge et de la longueur de l'élément de charge.
  \item En utilisant la symétrie de la situation, déduisez la valeur de la
    composante horizontale du champ électrique net.
  \item L'élément de charge est suffisamment petit pour se comporter comme une
    charge ponctuelle. Montrez que l'expression de la composante $y$ du champ
    électrique au point $P$ est
    \[
      dE_y = \frac{k \lambda dx}{r^2} \cos \theta.
    \]
  \item L'expression précédente contient trois variables $x$, $r$ et $\theta$.
    Pour être en mesure de calculer l'intégrale, vous devez exprimer le champ
    électrique en terme d'une seule variable. Ici, vous utiliserez $\theta$.
    Exprimez $r$ en fonction de $D$ et $\theta$. Exprimez $x$ en fonction de
    $\theta$ puis dérivez pour trouver $dx$ en fonction de $d\theta$
    \marginnote{Rappelez-vous que la dérivée de $\tan \theta$ est $\sec^2 \theta$}.
    Vous devriez obtenir
    \[
      dE_y = \frac{k \lambda d\theta}{D} \cos \theta.
    \]
  \item Appliquez le principe de superposition pour obtenir une intégrale
    donnant la composante $y$ du champ électrique net. Les bornes d'intégration
    devraient être choisies de telle sorte que vous couvrez la tige au complet.

  \begin{marginfigure}
    Indice pour les bornes d'intégration:

    \begin{tikzpicture}
      \draw (-3, 0) -- (3, 0);
      \draw (-3, -0.1) -- (3, -0.1);
      \coordinate (P) at (0, 3);
      \coordinate (O) at (0, 0);
      \coordinate (r) at (3, 0);
      \node[anchor=west] (Pn) at (P) {$P$};
      \fill (P) circle(2pt);
      \draw[<->] (O) -- node[fill=white] {$D$} ($(P) - (0, 0.08)$);
      \fill ($(r) + (0.1, 0)$) rectangle ++(-0.2, -0.1);
      \draw[dashed] (r) -- node[right] {$\sqrt{L^2/4 + D^2}$} (P);
      \draw ($(P) - (0, 0.5)$) arc (-90:{-atan(3/3}:0.5);
      \node at ($(P) + (0.2, -0.7)$) {$\theta$};
      \draw (0, -0.1) -- (0, -0.3) node[below] {$0$};
      \draw (r) -- ++(0, -0.3) node[below] {$L/2$};
    \end{tikzpicture}
  \end{marginfigure}

  \item Évaluez l'intégrale. Vous devriez trouver
    \[
      \vec{E} = \frac{k \lambda}{D} \frac{L}{\sqrt{L^2 / 4 + D^2}} \yhat.
    \]

  \item Montrez que lorsque la distance $D$ est beaucoup plus grande que la
    longueur de la tige, le champ électrique produit par la tige se rapproche
    de celui d'une charge ponctuelle :
    \[
      \vec{E} = \frac{k\lambda L}{D^2} \yhat.
    \]

  \item Montrez que lorsque la tige est très longue ($L$ est beaucoup plus
    grand que la distance $D$), le champ électrique produit par la tige est
    donné par
    \[
      \vec{E} = \frac{2k\lambda }{D} \yhat.
    \]

\end{enumerate}


\section{Champ électrique d'un arc uniformément chargé}

On considère un arc de cercle de rayon $R$ qui couvre un angle $\alpha$ et
porte une densité linéique de charge $\lambda$.  Quel est le champ électrique
au centre du cercle?

\begin{marginfigure}
  \begin{tikzpicture}
    \draw[decorate, decoration={
      markings, mark=between positions 0.1 and 1 step 6mm with {
        \node (0, 0) {$+$};
    }}] (-40:2.5) arc (-40:40:2.5);
    \draw (-40:2.7) arc (-40:40:2.7);
    \draw (-40:2.3) arc (-40:40:2.3);
    \draw[fill=black] (0, 0) circle (1pt);
    \draw (0, 0) -- node[fill=white] {$R$} ++(40:2.3);
    \draw (0, 0) -- ++(-40:2.3);
    \draw (-40:0.3) arc (-40:40:0.3);
    \draw[dashed, <-] (-1, 0) node[below] {$x$} -- (3, 0);
    \draw[fill=black!20] (-30:2.3) -- (-30:2.7) arc (-30:-25:2.7) -- (-25:2.3) -- cycle;
    \draw[dashed] (-27.5:2.3) -- (0, 0);
    \draw (0:1) arc (0:-27.5:1);
    \draw[->|] (-34:2.8) -- (-30:2.8);
    \draw[->|] (-21:2.8) -- (-25:2.8);
    \node at (-27:3.2) {$Rd\theta$};
    \node at (-14:1.3) {$\theta$};
    \node[fill=white] at (0:0.6) {$\alpha$};
  \end{tikzpicture}
\end{marginfigure}

\begin{enumerate}
  \item En utilisant un argument de symétrie, montrez que le champ électrique
    net n'a qu'une composante parallèle à l'axe des $x$.
  \item Montrez que la composante $x$ du champ électrique produit par un
    élément de charge $dq$ est donné par
    \[
      \dif E_x =  \frac{k\lambda R \ddif \theta}{R^2} \cos \theta.
    \]
  \item Appliquez le principe de superposition pour trouver le champ électrique
    résultant
    \[
      \vec{E} = \frac{2k\lambda}{R} \sin(\alpha / 2) \xhat.
    \]
  \item À partir du résultat précédent, montrez que le champ électrique au
    centre d'un anneau uniformément chargé est nul.
\end{enumerate}


\end{document}
