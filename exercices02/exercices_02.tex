\documentclass{tufte-handout}

\usepackage{cmbright}
\usepackage[french]{babel}
\usepackage[utf8]{inputenc}
\usepackage[T1]{fontenc}
\usepackage{amsmath, amsthm, amsfonts}
\usepackage[separate-uncertainty]{siunitx}
\usepackage{xcolor}
\usepackage{tikz}
\usepackage{pgfplots}
\usepackage{circuitikz}
\usepackage{hyperref}
\usepackage{booktabs}
\usepackage[version=3]{mhchem}
\pgfplotsset{compat = 1.3}

% Paul Tol's qualitative palette
% ``bright''.https://personal.sron.nl/~pault/#sec:qualitative
\definecolor{tblue}{HTML}{4477AA}
\definecolor{tcyan}{HTML}{66CCEE}
\definecolor{tgreen}{HTML}{228833}
\definecolor{tyellow}{HTML}{CCBB44}
\definecolor{tred}{HTML}{EE6677}
\definecolor{tpurple}{HTML}{AA3377}
\definecolor{tgrey}{HTML}{BBBBBB}


% Default arrow for tikz and style for positive and negative objects.
\tikzset{>=latex,
    negative/.style={draw=teal!70!black, fill=teal!10, thick},
    positive/.style={draw=red, fill=red!10, thick}}
\usetikzlibrary{matrix,calc,decorations.pathreplacing,decorations.pathmorphing,decorations.markings}

% French locale for numbers and negative exponent for units.
\sisetup{locale=FR, per-mode=symbol}

\newcommand{\abs}[1]{\left| #1 \right|}
\newcommand{\rhat}{\vec{\hat{r}}}
\newcommand{\xhat}{\vec{\imath}}
\newcommand{\yhat}{\vec{\jmath}}
\newcommand{\zhat}{\vec{k}}
\newcommand{\real}{\mathbb{R}}
\newcommand{\der}[2]{\frac{\mathrm{d}#1}{\mathrm{d}#2}}
\newcommand{\pder}[2]{\frac{\partial\ #1}{\partial\ #2}}
\newcommand{\dif}{\mathrm{d}}
\newcommand{\ddif}{\,\mathrm{d}}
\newcommand{\grad}{\vec{\nabla}}
\newcommand{\exemple}[1]{\begin{fullwidth}#1\end{fullwidth}}
\newcommand{\norm}[1]{\lVert\ #1\ \rVert}
\newcommand{\vu}{\vec{u}}
\newcommand{\vv}{\vec{v}}
\newcommand{\vr}{\vec{r}}
\newcommand{\va}{\vec{a}}
\newcommand{\vF}{\vec{F}}
\newcommand{\vE}{\vec{E}}
\newcommand{\vB}{\vec{B}}
\newcommand{\vecxyz}[3]{#1 \xhat\ + #2 \yhat\ + #3 \zhat}
\newcommand{\vecxy}[2]{#1 \xhat\ + #2 \yhat}
\newcommand{\coulombcst}{k}
\newcommand{\emf}{\ensuremath{\mathcal{E}}}
\newcommand{\eval}{\SI{1.602e-19}{C}}
\newcommand{\kval}{\SI{8.99e9}{Nm^2 \per C^2}}




% Pour cacher les réponses, utiliser la valeur 0
%\def\reponse{0}
% Pour afficher les réponses, utiliser la valeur 1
\def\reponse{0}


\title{Champ magnétique}
\date{}
\author{203-NYB-05 Électricité et magnétisme}


\begin{document}

\maketitle


\section{Champ d'un fil, le long de l'axe}

On cherche le champ magnétique d'un long fil droit parcouru d'un courant
constant $i$ sur l'axe du fil, au point $P$. À partir de la loi de Biot-Savart,
expliquez clairement pourquoi ce champ doit être nul.

\begin{marginfigure}
  \begin{center}
  \begin{tikzpicture}
    \draw[thick] (0, 0) -- (0, 3);
    \draw[very thick, ->] (0.2, 2.5) -- (0.2, 2.9) node[right] {$i$};
    \fill (0, -1) circle(2pt);
    \node at (0.3, -1) {$P$};
  \end{tikzpicture}
  \end{center}
\end{marginfigure}

\if\reponse1
  {\color{tblue}
  }
\else
  \vspace{5cm}
\fi


\section{Champ d'un arc de cercle}

On considère le fil en forme d'arc de cercle ci-contre qui sous-tend un angle
de \SI{60}{\degree} et qui a un rayon de \SI{8}{\centi\meter}. Le fil est
parcouru d'un courant de \SI{12}{\ampere} dans la direction indiquée sur le
schéma. À partir de la loi de Biot-Savart, déterminez le champ magnétique au
centre de l'arc de cercle.

\begin{marginfigure}
  \begin{center}
  \begin{tikzpicture}
    \draw[thick] (3, 0) arc (0:60:3);
    \draw[very thick, <-] (2.8579, 1.65) arc (30:10:3.3) node[right] {$i$};
    \fill (0, 0) circle(2pt);
    \node at (-0.2, -0.2) {$P$};
  \end{tikzpicture}
  \end{center}
\end{marginfigure}

\if\reponse1
  {\color{tblue}
  }
\else
  \vspace{4cm}
\fi

\clearpage

\section{Champ d'un ensemble de fils}

On considère les trois bouts de fil ci-contre. On a $r = \SI{3}{\centi\meter}$,
$i_1 = \SI{2}{\ampere}$, $i_2 = \SI{5}{\ampere}$, $i_3 = \SI{1}{\ampere}$. Les
fils portant les courants $i_2$ et $i_3$ sont de très longs fils
perpendiculaires au plan de la page. Déterminez le champ magnétique au point
$P$.

Vous pouvez utiliser les relations obtenues en classe pour le champ d'un long
fil et le champ d'un fil en forme d'arc de cercle sans les redémontrer
à partir de la loi de Biot-Savart.

\begin{marginfigure}
  \begin{center}
  \begin{tikzpicture}
    \coordinate (i3) at (-2, 0);
    \coordinate (i2) at (1, -2);
    \draw[thick] (0, 4) -- (0, 2) arc (90:0:2) -- (4, 0);
    \draw[very thick, ->] (1.99, 1.15) arc (30:10:2.3) node[right] {$i_1$};
    \fill (0, 0) circle(2pt);
    \node at (-0.2, -0.2) {$P$};
    \draw[densely dashed, <->] (0, 0) -- node[fill=white] {$r$} ++(90:2);
    \draw[densely dashed, <->] (0, 0) -- node[fill=white] {$r$} ++(0:2);
    \draw[densely dashed] (-2, 0) -- node[fill=white] {$r$} (0, 0);
    \draw[densely dashed] (0, 0) -- node[fill=white] {$r$} (0, -2)
                          -- node[fill=white] {$\frac{r}{2}$} (i2);
    \draw[thick] (i2) circle(3pt);
    \fill (i2) circle(1pt);
    \node[anchor=north west] at (i2) {$i_2$};
    \draw[thick, fill=white] (i3) circle(3pt);
    \node at (i3) {$\times$};
    \node[anchor=north east] at (i3) {$i_3$};
  \end{tikzpicture}
  \end{center}
\end{marginfigure}

\end{document}
