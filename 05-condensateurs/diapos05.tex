\documentclass{beamer}
\beamertemplatenavigationsymbolsempty
\usepackage[french]{babel}
\usepackage{fontspec}
\usepackage{amsmath, amsthm, amsfonts}
\usepackage[separate-uncertainty]{siunitx}
\usepackage{xcolor}
\usepackage{tikz}
\usepackage{tikz-cd}
\usepackage[object=vectorian]{pgfornament}
\usepackage{circuitikz}
\usepackage{hyperref}
\usepackage{caption}
\usepackage{booktabs}
\usepackage{mathtools}
\usepackage{longtable}
\usepackage[version=3]{mhchem}

\tikzset{>=latex,
    negative/.style={draw=teal!70!black, fill=teal!10, thick},
    positive/.style={draw=red, fill=red!10, thick}}
\usetikzlibrary{matrix,calc,decorations.pathreplacing,decorations.pathmorphing,decorations.markings}
\sisetup{locale=FR, per-mode=symbol}

\newcommand{\abs}[1]{\left| #1 \right|}
\newcommand{\rhat}{\vec{\hat{r}}}
\newcommand{\xhat}{\vec{\imath}}
\newcommand{\yhat}{\vec{\jmath}}
\newcommand{\zhat}{\vec{k}}
\newcommand{\real}{\mathbb{R}}
\newcommand{\der}[2]{\frac{\mathrm{d}#1}{\mathrm{d}#2}}
\newcommand{\pder}[2]{\frac{\partial\ #1}{\partial\ #2}}
\newcommand{\dif}{\mathrm{d}}
\newcommand{\ddif}{\,\mathrm{d}}
\newcommand{\grad}{\vec{\nabla}}
\newcommand{\exemple}[1]{\begin{fullwidth}#1\end{fullwidth}}
\newcommand{\norm}[1]{\lVert\ #1\ \rVert}
\newcommand{\vu}{\vec{u}}
\newcommand{\vv}{\vec{v}}
\newcommand{\vr}{\vec{r}}
\newcommand{\va}{\vec{a}}
\newcommand{\vF}{\vec{F}}
\newcommand{\vE}{\vec{E}}
\newcommand{\vB}{\vec{B}}
\newcommand{\vecxyz}[3]{#1 \xhat\ + #2 \yhat\ + #3 \zhat}
\newcommand{\vecxy}[2]{#1 \xhat\ + #2 \yhat}
\newcommand{\coulombcst}{k}
\newcommand{\emf}{\ensuremath{\mathcal{E}}}

% Nice separator line
\newcommand{\sectionline}{
    \noindent
    \begin{center}
        \resizebox{0.5\linewidth}{1ex}
    {{%
    {\begin{tikzpicture}
    \node  (C) at (0,0) {};
    \node (D) at (9,0) {};
    \path (C) to [ornament=85] (D);
    \end{tikzpicture}}}}
    \end{center}
}

\theoremstyle{definition}
\newtheorem*{defn}{Definition}


\usepackage[version=3]{mhchem}

\setbeamercolor{title}{fg=tblue}
\setbeamercolor{frametitle}{fg=tblue}
\setbeamercolor{structure}{fg=tblue}

% Make footnotesize smaller
\makeatletter
\renewcommand\footnotesize{%
   \@setfontsize\footnotesize\@viipt{11}%
   \abovedisplayskip 8\p@ \@plus2\p@ \@minus4\p@
   \abovedisplayshortskip \z@ \@plus\p@
   \belowdisplayshortskip 4\p@ \@plus2\p@ \@minus2\p@
   \def\@listi{\leftmargin\leftmargini
               \topsep 4\p@ \@plus2\p@ \@minus2\p@
               \parsep 2\p@ \@plus\p@ \@minus\p@
               \itemsep \parsep}%
   \belowdisplayskip \abovedisplayskip
}
\makeatother

\title{Électricité et magnétisme}
\subtitle{Chapitre 5 - Condensateurs}
\date{29 septembre 2021}
\author{Loïc Séguin-Charbonneau}
\institute{Cégep Édouard-Montpetit}

\begin{document}

\maketitle


\begin{frame}{Deux grandes plaques parallèles}

On considère deux grandes plaques parallèles connectées à une source de
tension. Chacune des plaques a une surface $A$. Si on applique une différence
de potentiel $\Delta V$ entre les deux plaques, quelle charge sera emmagasinée
sur chacune des plaques?

\begin{center}
  \includegraphics[scale=0.4]{figures/condensateur-plan.pdf}
\end{center}

\end{frame}


\begin{frame}[t]{Deux grande plaques parallèles}

On considère deux grandes plaques parallèles connectées à une source de
tension. Chacune des plaques a une surface $A$. Si on applique une différence
de potentiel $\Delta V$ entre les deux plaques.

\begin{center}
  \includegraphics[scale=0.4]{figures/condensateur-plan.pdf}
\end{center}


\only<1-2>{
  Quelle expression décrit le champ électrique entre les plaques?

  \begin{enumerate}[A.]
    \item $\vE = \frac{\sigma}{\varepsilon_0}$ vers le haut
    \item<alert@2> $\vE = \frac{\sigma}{\varepsilon_0}$ vers le bas
    \item $\vE = \frac{\sigma}{2\varepsilon_0}$ vers le haut
    \item $\vE = \frac{\sigma}{2\varepsilon_0}$ vers le bas
  \end{enumerate}
}

\only<3-4>{
  Si la charge accumulée sur la plaque positive est $Q$, quelle expression
  décrit la grandeur du champ électrique entre les plaques?

  \begin{enumerate}[A.]
    \item<alert@4> $E = \frac{Q}{A\varepsilon_0}$
    \item $E = \frac{2Q}{A\varepsilon_0}$
    \item $E = \frac{A}{Q\varepsilon_0}$
    \item $E = \frac{A}{2Q\varepsilon_0}$
  \end{enumerate}
}

\only<5-6>{
  Quelle est la différence de potentiel si on passe de la plaque négative à la
  plaque positive?

  \begin{enumerate}[A.]
    \item $\Delta V = \frac{-Q}{dA\varepsilon_0}$
    \item $\Delta V = \frac{Q}{dA\varepsilon_0}$
    \item $\Delta V = \frac{-Qd}{A\varepsilon_0}$
    \item<alert@6> $\Delta V = \frac{Qd}{A\varepsilon_0}$
  \end{enumerate}
}

\only<7-8>{
  Quel est le lien entre la charge emmagasinée sur la plaque positive et la
  différence de potentiel?

  \begin{enumerate}[A.]
    \item $Q = \frac{A}{d\varepsilon_0} \Delta V$
    \item $Q = \frac{\varepsilon_0 d}{A} \Delta V$
    \item<alert@8> $Q = \frac{\varepsilon_0 A}{d} \Delta V$
    \item $Q = \frac{d\varepsilon_0}{A} \Delta V$
  \end{enumerate}
}

\end{frame}


\begin{frame}{Capacité}
  Un condensateur de \SI{50}{\micro\farad} est connecté à une source de tension
  de \SI{10}{\volt}. Quelle est la charge sur son armature positive?

  \begin{enumerate}[A.]
    \item \SI{0.2}{\micro\coulomb}
    \item \SI{50}{\micro\coulomb}
    \item<alert@2> \SI{500}{\micro\coulomb}
    \item \SI{2e5}{\coulomb}
  \end{enumerate}
\end{frame}

\begin{frame}[t]{Propriétés du condensateur plan}

Lequel des énoncés suivants est vrai.

\begin{enumerate}[A.]
  \item Si l'aire des plaques augmente, la capacité diminue.
  \item<alert@2> La charge accumulée sur un condensateur plan augmente
    proportionnellement à la différence de potentiel entre les plaques.
  \item Si on augmente la distance entre les plaques, l'énergie potentielle du
    condensateur diminue.
\end{enumerate}
\end{frame}


\begin{frame}[t]{Propriétés du condensateur plan}

Lesquels des énoncés suivants sont vrais.

\begin{enumerate}[A.]
  \item<alert@2> Plus la distance entre les plaques est grande, plus la
    différence de potentiel doit être élevée pour maintenir la même charge sur
    les plaques.
  \item<alert@2> Si la densité surfacique de charge augmente et que la
    différence de potentiel demeure la même, c'est parce que la distance entre
    les plaques a diminué.
  \item Pour un condensateur donné, plus la différence de potentiel entre les
    plaques augmente, plus la capacité diminue.
\end{enumerate}

\end{frame}


\begin{frame}{Rigidité diélectrique}

\begin{center}
\begin{tabular}{lS}
  \toprule
  Substance       &        {Rigidité diélectrique (\si{V/cm})}     \\
  \midrule
  Air             &  30000  \\
  Verre           &  100000 \\
  Polystyrène     &  197000 \\
  Papier ciré     &  500000 \\
  Diamant         &  20000000 \\
  \bottomrule
\end{tabular}
\end{center}
\end{frame}



\begin{frame}{Explosion de condensateurs}
  \includegraphics[width=\textwidth]{figures/explosion-condensateur.png}
  \url{https://youtu.be/XBoaBwMRbnk?t=30}
\end{frame}


\begin{frame}{Explosion de condensateurs}

Dans un des cas, on voit un condensateur de \SI{470}{\micro\farad} qui explose.
Supposons qu'il a explosé à une tension de \SI{200}{\volt}. Quelle est la
quantité d'énergie qui peut être relâchée durant cette explosion?

\end{frame}



\begin{frame}{Exercice}

On a deux condensateurs identiques. Le condensateur A porte une charge de
\SI{100}{\micro\coulomb} et le condensateur B porte une charge de
\SI{50}{\micro\coulomb}. Quel énoncé est vrai?

\begin{enumerate}[A.]
  \item La capacité du condensateur A est deux fois plus grande que celle du
    condensateur B.
  \item<alert@2> Le champ électrique entre les armatures du condensateur A est
    deux fois plus grand que celui de B.
  \item La différence de potentiel du condensateur A est deux fois plus petite
    que celle de B.
  \item Il y a deux fois plus d'énergie emmagasinée dans le condensateur A que
    le B.
\end{enumerate}

\end{frame}



\begin{frame}{Exercice}

On a deux condensateurs identiques sauf pour le diélectrique. Dans le
condensateur A, le diélectrique est du papier (constante diélectrique de 3).
Dans le condensateur B le diélectrique est du mica (constante diélectrique de
6). Les deux condensateurs portent la même charge. Quels énoncés sont vrais.

\begin{enumerate}[A.]
  \item La capacité du condensateur A est deux fois plus grande que celle du
    condensateur B.
  \item<alert@2>  Le champ électrique entre les armatures du condensateur A est
    deux fois plus grand que celui de B.
  \item La différence de potentiel du condensateur A est deux fois plus petite
    que celle de B.
  \item<alert@2> Il y a deux fois plus d'énergie emmagasinée dans le
    condensateur A que le B.
\end{enumerate}

\end{frame}


\begin{frame}{Exercice circuit avec condensateur}

On construit le circuit suivant avec une pile de \SI{9}{V}. Le condensateur 1 a
une capacité $C_1 = \SI{45}{\micro\farad}$ et ses armatures sont séparées par
du vide. Les condensateurs 2 et 3 sont construits de la même façon que le
condensateur 1 sauf que l'espace entre leurs armatures est rempli par
du germanium et du papier, respectivement. La constante diélectrique du
germanium est 16 alors que celle du papier est 3.

\begin{center}
\begin{circuitikz}[yscale=0.7]
  % French babel breaks everything... see https://latex.org/forum/viewtopic.php?t=11981
  \shorthandoff{:}\shorthandoff{!}
  \draw (0, 0) to[battery, l=$\SI{9}{\volt}$] (0, 3)
    to[C, l=$C_1$] (2, 3)
    to[C, l=$C_2$] (2, 0)
    to (0, 0);
  \draw (2, 3) to[short] (4, 3)
    to[C, l=$C_3$] (4, 0)
    to (2, 0);
\end{circuitikz}
\end{center}

\begin{enumerate}
  \item Déterminer la capacité équivalente à ces trois condensateurs.
  \item Déterminer la charge accumulée sur la plaque positive du condensateur 2.
  \item Déterminer l'énergie accumulée dans le condensateur 3.
\end{enumerate}

\end{frame}

\end{document}
