\documentclass{beamer}
\beamertemplatenavigationsymbolsempty
\usepackage[french]{babel}
\usepackage{fontspec}
\usepackage{amsmath, amsthm, amsfonts}
\usepackage[separate-uncertainty]{siunitx}
\usepackage{xcolor}
\usepackage{tikz}
\usepackage{tikz-cd}
\usepackage[object=vectorian]{pgfornament}
\usepackage{circuitikz}
\usepackage{hyperref}
\usepackage{caption}
\usepackage{booktabs}
\usepackage{mathtools}
\usepackage{longtable}
\usepackage[version=3]{mhchem}

\tikzset{>=latex,
    negative/.style={draw=teal!70!black, fill=teal!10, thick},
    positive/.style={draw=red, fill=red!10, thick}}
\usetikzlibrary{matrix,calc,decorations.pathreplacing,decorations.pathmorphing,decorations.markings}
\sisetup{locale=FR, per-mode=symbol}

\newcommand{\abs}[1]{\left| #1 \right|}
\newcommand{\rhat}{\vec{\hat{r}}}
\newcommand{\xhat}{\vec{\imath}}
\newcommand{\yhat}{\vec{\jmath}}
\newcommand{\zhat}{\vec{k}}
\newcommand{\real}{\mathbb{R}}
\newcommand{\der}[2]{\frac{\mathrm{d}#1}{\mathrm{d}#2}}
\newcommand{\pder}[2]{\frac{\partial\ #1}{\partial\ #2}}
\newcommand{\dif}{\mathrm{d}}
\newcommand{\ddif}{\,\mathrm{d}}
\newcommand{\grad}{\vec{\nabla}}
\newcommand{\exemple}[1]{\begin{fullwidth}#1\end{fullwidth}}
\newcommand{\norm}[1]{\lVert\ #1\ \rVert}
\newcommand{\vu}{\vec{u}}
\newcommand{\vv}{\vec{v}}
\newcommand{\vr}{\vec{r}}
\newcommand{\va}{\vec{a}}
\newcommand{\vF}{\vec{F}}
\newcommand{\vE}{\vec{E}}
\newcommand{\vB}{\vec{B}}
\newcommand{\vecxyz}[3]{#1 \xhat\ + #2 \yhat\ + #3 \zhat}
\newcommand{\vecxy}[2]{#1 \xhat\ + #2 \yhat}
\newcommand{\coulombcst}{k}
\newcommand{\emf}{\ensuremath{\mathcal{E}}}

% Nice separator line
\newcommand{\sectionline}{
    \noindent
    \begin{center}
        \resizebox{0.5\linewidth}{1ex}
    {{%
    {\begin{tikzpicture}
    \node  (C) at (0,0) {};
    \node (D) at (9,0) {};
    \path (C) to [ornament=85] (D);
    \end{tikzpicture}}}}
    \end{center}
}

\theoremstyle{definition}
\newtheorem*{defn}{Definition}


\usepackage[version=3]{mhchem}

\setbeamercolor{title}{fg=tblue}
\setbeamercolor{frametitle}{fg=tblue}
\setbeamercolor{structure}{fg=tblue}

% Make footnotesize smaller
\makeatletter
\renewcommand\footnotesize{%
   \@setfontsize\footnotesize\@viipt{11}%
   \abovedisplayskip 8\p@ \@plus2\p@ \@minus4\p@
   \abovedisplayshortskip \z@ \@plus\p@
   \belowdisplayshortskip 4\p@ \@plus2\p@ \@minus2\p@
   \def\@listi{\leftmargin\leftmargini
               \topsep 4\p@ \@plus2\p@ \@minus2\p@
               \parsep 2\p@ \@plus\p@ \@minus\p@
               \itemsep \parsep}%
   \belowdisplayskip \abovedisplayskip
}
\makeatother

\title{Électricité et magnétisme}
\subtitle{Chapitre 6 - Courant électrique}
\date{5 octobre 2021}
\author{Loïc Séguin-Charbonneau}
\institute{Cégep Édouard-Montpetit}

\begin{document}

\maketitle


\begin{frame}{Manic-5}
  \includegraphics[width=\textwidth]{figures/Manic-5.jpg}
\end{frame}


\begin{frame}{Transport d'électricité}
Un cable de transport d'électricité à haute tension d'Hydro-Québec fait de
cuivre et d'un diamètre d'environ \SI{10}{cm} relie la centrale Manic-5 à
Montréal et a une longueur de \SI{685}{km}. La densité de courant dans le câble
est de \SI{0.8}{A/mm^2}.

\begin{enumerate}
  \item Quel est le courant qui circule dans le câble?
  \item Combien d'électrons traversent une section du câble à chaque minute?
  \item Déterminez la résistance du câble (la résistivité du cuivre est de
    \SI{1.678e-8}{\ohm\meter}).
  \item Quel est le champ électrique dans le câble?
  \item En combien de temps un électron partant de Manic-5 atteindrait-il
    Montréal? (La
    mobilité des électrons dans le cuivre est de \SI{0.0033}{m^2/Vs}.)
  \item Combien d'énergie est perdue sous forme de chaleur dans le câble à
    chaque jour?
  \item Sachant que la centrale produit une puissance de \SI{1596}{MW}, quelle
    est la proportion de la puissance produite qui est perdue dans le câble?
\end{enumerate}
\end{frame}

\end{document}
